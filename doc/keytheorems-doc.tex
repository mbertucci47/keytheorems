% !TeX program = txs:///lualatex | txs:///makeindex | txs:///lualatex | txs:///view-pdf
\PassOptionsToPackage{hyperindex}{hyperref}
\PassOptionsToPackage{dvipsnames}{xcolor}
\documentclass{ltxdoc}
\usepackage{geometry}
\geometry{margin=1.5in}
\usepackage{tcolorbox}
\tcbuselibrary{documentation,minted,breakable,hooks}
\tcbset{
  listing engine=minted,
  minted language=latex,
  }
\ifTUTeX
\usepackage{newcomputermodern}
\else
\usepackage{lmodern}
\usepackage[T1]{fontenc}
\usepackage{amssymb}
\fi
\usetikzlibrary{cd}
\usepackage{booktabs}
\usepackage{keytheorems}
\usepackage{cleveref}

%%% TEMPORARY (https://tex.stackexchange.com/q/730126/208544)
\RemoveFromHook{label}[firstaid/cleveref]
\makeatletter
\AddToHook{label}[firstaid/cleveref]
         {\ifx
           \@currentcounter\@empty
          \else
           \firstaid@cref@updatelabeldata{\@currentcounter}%<--- was missing
          \fi} 
\makeatother
%%% END TEMPORARY

\makeatletter
\newcommand\keythmsversion{\@keythms@version}
\makeatother

\title{%
  \pkg{keytheorems} package \\[1ex]
  \large version \keythmsversion \\[1ex]
  \href{https://github.com/mbertucci47/keytheorems}
    {\texttt{github.com/mbertucci47/keytheorems}}
  }
\author{Matthew Bertucci}

\newkeytheoremstyle{breaksty}{break}
\newkeytheorem{theorem}
\newkeytheorem{mythm}[name=Some Name]
\newkeytheorem{theorem*}[name=Theorem,numbered=false]
\newkeytheorem{conjecture}[parent=section]
\newkeytheorem{lemma}[sibling=theorem]
\newkeytheorem{test}[
  preheadhook=PREHEAD,
  postheadhook=POSTHEAD,
  prefoothook=PREFOOT,
  postfoothook=POSTFOOT
]
\newkeytheorem{prop}[
  name=Proposition,
  refname={proposition,propositions},
  Refname={Proposition,Propositions}
  ]
\newkeytheorem{example}[qed]
\newkeytheorem{solution}[qed=$\clubsuit$]
\newkeytheorem{remark}[style=remark]
\tcbset{
  defstyle/.style={arc=0mm,colback=blue!5!white,colframe=blue!75!black},
  thmstyle/.style={colback=green!5,colframe=green!35!black},
  }
\newkeytheoremstyle{tcb-standard}{
  tcolorbox=thmstyle,
  headpunct={},
  notebraces={}{},
  noteseparator={: },
  notefont=\bfseries,
  bodyfont=\normalfont,
  }
\newkeytheorem{mytheo}[name=My Theorem,style=tcb-standard]
\newkeytheorem{corollary}[tcolorbox]
\newkeytheorem{definition}[
  style=definition,
  tcolorbox={defstyle}
  ]
\newkeytheorem{boxcor}[
  tcolorbox-no-titlebar={colback=red!10},
  name=Corollary,
  sibling=corollary,
  ]
\newkeytheorem{observation}[style=breaksty]
\newkeytheorem{quotethm}[name=Quote Theorem,leftmargin=1cm,rightmargin=1cm]
\newkeytheorem{indentedthm}[name=Indented Theorem,leftmargin=1cm]

\MakeShortVerb{\|}
\newcommand{\env}{\texttt}
\newcommand{\hook}{\texttt}
\NewCommandCopy{\braces}{\brackets}
\newcommand{\bracks}[1]{\texttt{[#1]}}
\newcommand{\ttbraces}[1]{\braces{\texttt{#1}}}
%\tcbset{index gather commands=false}
\usepackage{xcolor-solarized}
\colorlet{ExampleBack}{solarized-base3!50}
\colorlet{ExampleBackLower}{solarized-base3}
\tcbset{
  color command=BrickRed,
  color key=Fuchsia,
  color value=teal,
  before doc body={%
    \setlength{\parindent}{15pt}
    \noindent
    },
  docexample/.append style={boxrule=0pt,arc=0pt,after app={\noindent}}
  }
\tcbset{
  withpreamble/.style = {
    before upper = {%
      {\ttfamily\itshape \% preamble}%
      \vspace{-0.7\baselineskip}%
      \tcbusetemplisting
      \noindent{\ttfamily\itshape \% document}%
      \vspace{-0.7\baselineskip}%
      }
    }
  }

\NewTCBListing{keythmscode}{ O{} }
  {
    colback=ExampleBack,
    arc=0pt,
    boxrule=0pt,
%    listing side text,
%    lower separated=false,
    skin=bicolor,
    colbacklower=ExampleBackLower,
    breakable,
    parbox=false, % so examples match document with indent, etc.
%    enhanced,
%    segmentation engine=path,
%    segmentation style={solid,gray},
    #1
  }
\newtcolorbox{notebox}[1][]{enhanced,
  before skip balanced=2mm,after skip balanced=4mm,
  boxrule=0.4pt,left=5mm,right=2mm,top=1mm,bottom=1mm,
  colback=red!20,
  colframe=red!40!black,
  sharp corners,rounded corners=southeast,arc is angular,arc=3mm,
  underlay={%
    \path[fill=tcbcolback!80!black] ([yshift=3mm]interior.south east)--++(-0.4,-0.1)--++(0.1,-0.2);
    \path[draw=tcbcolframe,shorten <=-0.05mm,shorten >=-0.05mm] ([yshift=3mm]interior.south east)--++(-0.4,-0.1)--++(0.1,-0.2);
    \path[fill=red!40!black,draw=none] (interior.south west) rectangle node[white]{\LARGE\bfseries !} ([xshift=4mm]interior.north west);
    },
  drop fuzzy shadow,#1}
\makeatletter
% allow for optional argument specifying printed text, mostly for starred keys
\RenewDocumentCommand\tcb@ref@doc{msom}{%
  \hyperref[{#1:#4}]{%
  \IfValueTF{#3}{\texttt{#3}}{\texttt{\ref*{#1:#4}}}%
  \IfBooleanTF{#2}{}{%
    \ifnum\getpagerefnumber{#1:#4}=\thepage\relax%
    \else%
      \kvtcb@doc@format@page{{\fontfamily{pzd}\fontencoding{U}\fontseries{m}\fontshape{n}\selectfont\char213}%
      \,\kvtcb@text@pageshort\,\pageref*{#1:#4}}%
    \fi}}%
}
\makeatother
\makeindex

\begin{document}

\maketitle

\begin{abstract}
An \pkg{expl3}-implementation of a key-value interface to \pkg{amsthm}, implementing most of the functionality provided by \pkg{thmtools}.
Several issues encountered with \pkg{thmtools} are avoided (see the README for a list) and a few new features are added.
\end{abstract}

\tableofcontents

\section{Dependencies}

The package depends on the \pkg{aliascnt}, \pkg{amsthm}, \pkg{etoolbox}, \pkg{refcount}, \pkg{translations} packages.
The \refKey{tcolorbox} and \refKey{tcolorbox-no-titlebar} keys require \pkg{tcolorbox}, and the \refKey[numbered=unless-unique]{numbered} key requires the \pkg{unique} package.
A \LaTeX{} kernel no older than 2023-06-01 is required; if older than 2024-06-01, \pkg{nameref} is required.
    
\section{Global options}

\begin{docCommand}{keytheoremset}
  {\marg{options}}
Every key in this section can be given as an option to \cs{usepackage} or in \cs{keytheoremset}, with the exception that \refKey{continues-code} can only be used in the latter.
\end{docCommand}

\subsection{Compatibility options}

\begin{docKey}{overload}
  {}
  {initially unset}
Redefines \cs{newtheorem} to internally use the \pkg{keytheorems} machinery. The syntax remains the same.
This is automatically set by \refKey{thmtools-compat}.
\end{docKey}

\begin{docKey}{thmtools-compat}
  {}
  {initially unset}
For compatibility with \pkg{thmtools} syntax.
For most documents,
\begin{dispListing}
\usepackage[thmtools-compat]{keytheorems}
\end{dispListing}
should be a drop-in replacement for \cs{usepackage}\ttbraces{amsthm,thmtools}.
The option defines the commands in the left column below.
The right column lists the corresponding \pkg{keytheorems} replacement that should be used in new documents.
\begin{center}
\begin{tabular}{rcl}
\multicolumn{1}{c}{\pkg{thmtools} command} & & \multicolumn{1}{c}{\pkg{keytheorems} replacement} \\
\toprule
\docAuxCommand*{declaretheorem} & $\rightarrow$ & \refCom{newkeytheorem} \\
\docAuxCommand*{declaretheoremstyle} & $\rightarrow$ & \refCom{newkeytheoremstyle} \\
\docAuxCommand*{listoftheorems} & $\rightarrow$ & \refCom{listofkeytheorems} \\
\docAuxCommand*{listtheoremname} & $\rightarrow$ & \refKey{title} key \\
\begin{tabular}{r@{}}
\docAuxCommand*{addtotheorempreheadhook}\\
\docAuxCommand*{addtotheorempostheadhook}\\
\docAuxCommand*{addtotheoremprefoothook}\\
\docAuxCommand*{addtotheorempostfoothook}\\
\end{tabular} & $\rightarrow$ & \refCom{addtotheoremhook} \\
\docAuxEnvironment*{restatable} environment & $\rightarrow$ & \refKey{store} key \\
\docAuxEnvironment*{restatable*} environment & $\rightarrow$ & \refKey[store*]{store} key
\end{tabular}
\end{center}
Also defined are the \docAuxKey*{shaded} and \docAuxKey*{thmbox} keys, implemented internally with \pkg{tcolorbox} rather than the \pkg{shadethm} and \pkg{thmbox} packages, respectively.
\end{docKey}

\subsection{Other global options}

\begin{docKey}{auto-translate}
  {\colOpt{=true\textbar false}}
  {default |true|, initially |true|}
If |false|, \pkg{keytheorems} does not automatically translate the title text used for \refCom{listofkeytheorems} and the note produced by the \refKey{continues} key.
These texts can be manually customized with the \refKey{title} and \refKey{continues-code} keys, respectively.
\end{docKey}

\begin{docKey}{continues-code}
  {=\meta{code with \textup{\texttt{\#1}}}}
  {initially\\ \cs{GetTranslation}\ttbraces{keythms\string_continues}\cs{pageref}\ttbraces{\#1}}
The code used to typeset the note produced by the \refKey{continues} key.
If English or an unknown language is used, defaults to \texttt{continuing from p.}\cs{,}\cs{pageref}\ttbraces{\#1}.
Currently (likely inaccurate!) translations exist for French, German, Italian, Portuguese, and Spanish.
\end{docKey}

\begin{docKey}{qed-symbol}
  {=\meta{symbol}}
  {initially \cs{openbox}}
Redefines \cs{qedsymbol} to be \meta{symbol}.
\end{docKey}

\begin{docKey}{restate-counters}
  {=\marg{comma-list of counters}}
  {initially \ttbraces{equation}}
Additional counters whose values are preserved when a theorem is restated.
This key does not reset the list, so you don't need to include |equation| in \meta{comma-list}.
\end{docKey}

\begin{docKey}{store-all}
  {}
  {initially unset}
Tells \pkg{keytheorems} to grab the body of each theorem so it can later be printed with the \refKey{print-body} option of \refCom{listofkeytheorems}.
Note that this means a theorem body \emph{cannot} contain verbatim material.
\end{docKey}

\begin{docKey}{store-sets-label}
  {}
  {initially unset}
Defines the \refKey{store} key to also set \refKey{label}, i.e. it makes |store=|\meta{tag} equivalent to |store=|\meta{tag}|,label=|\meta{tag}. Similarly for \refKey[store*]{store}.
\end{docKey}

\section{Defining theorems}

\begin{docCommand}{newkeytheorem}
  {\marg{env name}\oarg{options}}
Defines a theorem environment \meta{env name} which itself takes a few options (see \autoref{in-doc-keys}).
You can also declare multiple theorems at once by replacing \meta{env name} with a comma-list of names, e.g.
\begin{center}
\cs{newkeytheorem}\ttbraces{theorem,lemma,proposition}\bracks{\meta{options}}.
\end{center}

By default, the theorem's printed name is a title-cased \meta{env name}.
This can be changed with the \refKey{thm/name} key.
All \meta{options} are described in subsections \ref{thm-thmtools-keys} and \ref{thm-added-keys}.

\begin{tcbwritetemp}
\newkeytheorem{theorem}
\end{tcbwritetemp}

\begin{keythmscode}[withpreamble]
\begin{theorem}
Some text
\end{theorem}
\end{keythmscode}

\end{docCommand}

\begin{docCommands}[
  doc parameter=\marg{env name}\oarg{options}
  ]
  {
    {doc name = renewkeytheorem},
    {doc name = providekeytheorem},
    {doc name = declarekeytheorem},
  }
Sometimes a package or class defines theorems that need to be overwritten by the user. For this case, \pkg{keytheorems} provides \cs{renewkeytheorem} which redefines \meta{env name} or errors if it is not defined. For completeness, also provided are \cs{providekeytheorem} and \cs{declarekeytheorem}. The former only defines \meta{env name} if it is not already defined; the latter always overwrites \meta{env name}.
\end{docCommands}

\subsection{Keys available to theorem environments} \label{in-doc-keys}

As in \pkg{amsthm}, theorems can take an optional argument that contains a note or heading.

\begin{keythmscode}[]
\begin{theorem}[some heading]
Some text
\end{theorem}
\end{keythmscode}

Alternatively, the optional argument may contain any of the following keys.

\begin{docKey}{note}
  {=\meta{text}}
  {initially unset}
Alias \docAuxKey[][doc label=thmuse/name]{name}.
This is the key-value equivalent of the optional argument described above.
This syntax, however, allows the argument to contain other keys.

\begin{keythmscode}[]
\begin{theorem}[note=another heading]
Some more text
\end{theorem}
\end{keythmscode}

\end{docKey}

\begin{docKey}{short-note}
  {=\meta{text}}
  {initially unset}
Alias \docAuxKey{short-name}.
This replaces the value of \refKey{note} when displayed in the list of theorems (\refCom{listofkeytheorems}).
\end{docKey}

\begin{docKey}{label}
  {=\meta{label name}}
  {initially unset}
This is the key-value equivalent of |\begin{theorem}| \cs{label}\marg{label name}.

\begin{keythmscode}[]
\begin{theorem}[label=foo]
Some text
\end{theorem}
\ref{foo}
\end{keythmscode}

\end{docKey}

\begin{docKey}{continues}
  {\sarg=\meta{label name}}
  {initially unset}
Pick up a theorem where you left off.
The theorem number remains the same.
The printed text can be customized with the \refKey{continues-code} option.
The starred version also copies the theorem note, if it exists.

\begin{keythmscode}[]
\begin{theorem}[continues=foo]
\dots and some more text.
\end{theorem}
\end{keythmscode}

\end{docKey}

\begin{docKey}{store}
  {\sarg=\meta{tag}}
  {initially unset}
Alias \docAuxKey{restate}\sarg.
Stores the the theorem to be restated at any point in the document with \refCom{getkeytheorem}.
With the starred version, counters and labels are taken from the copy called with \cs{getkeytheorem}, so in this case can only be restated once.
This allows you, for example, to write all theorems and proofs in the appendix and call \cs{getkeytheorem} at the appropriate time mid-document.
For the numbering to be correct, the unstarred key will need at most two runs and the starred key at most three runs.

\begin{keythmscode}[]
\begin{theorem}[store=blub]
A theorem worth restating.
\end{theorem}
More brilliant mathematics.
\getkeytheorem{blub}
\end{keythmscode}

A theorem given this key \emph{cannot} contain verbatim material or other unexpected catcodes, such as a \pkg{tikz-cd} diagram.
The latter issue can be averted with the \texttt{ampersand-replacement} key.

\begin{tcbwritetemp}
\usepackage{tikz}
\usetikzlibrary{cd}
\end{tcbwritetemp}

\begin{keythmscode}[withpreamble]
\begin{lemma}[store=diagram]
Some commutative diagram:
\[\begin{tikzcd}[ampersand replacement=\&]
X\times_S Y \ar[r] \ar[d] \& X \ar[d] \\
Y \ar[r] \& S
\end{tikzcd}\]
\end{lemma}
\dots
\getkeytheorem{diagram}
\end{keythmscode}

\end{docKey}

\begin{docKey}{restate-keys}
  {=\marg{list of keys}}
  {initially unset}
Allows passing different keys to the restated theorem.
At the moment this is only useful with the \refKey{note} key.

\begin{keythmscode}[]
\begin{theorem}[
  store=rktest,
  note=ORIGINAL,
  restate-keys={note=RESTATED}]
Wow, yet another theorem.
\end{theorem}
\getkeytheorem{rktest}
\end{keythmscode}

\end{docKey}

\begin{docKey}{listhack}
  {=\docValue*{true}\textbar\docValue*{false}}
  {initially |false|}
Meant only to be used with the \refKey{break} style key for a theorem starting with a list.
Compare:

\begin{tcbwritetemp}
\newkeytheoremstyle{breaksty}{break}
\newkeytheorem{observation}[style=breaksty]
\end{tcbwritetemp}

\begin{keythmscode}[withpreamble]
\begin{observation}
\begin{enumerate}
\item First item
\end{enumerate}
\end{observation}

\begin{observation}[listhack=true]
\begin{enumerate}
\item First item
\end{enumerate}
\end{observation}
\end{keythmscode}

Note that the value |true| must be explicitly set so that |listhack| is not interpreted as the note text.
\end{docKey}

\begin{docKey}[][doc label=thm/seq]{seq}
  {=\meta{name}}
  {initially unset}
Adds the theorem to a custom sequence \meta{name} that can then be listed with \cs{listofkeytheorems}\bracks{seq=\meta{name}}.
See \refKey{listof/seq} for more details.
\end{docKey}

\subsection{Keys also defined in \pkg{thmtools}} \label{thm-thmtools-keys}

These are the \oarg{options} available to \cs{newkeytheorem}.
Except for \refKey{thm/name} and \refKey{style}, each key below can also be used in \refCom{newkeytheoremstyle}.
For more description, see the \href{https://ctan.org/pkg/thmtools}{\pkg{thmtools}} package.

\begin{docKey}[][doc label=thm/name]{name}
  {=\meta{display name}}
  {initially title-cased \meta{env name}}
Aliases \docAuxKey{heading} and \docAuxKey{title}.

\begin{tcbwritetemp}
\newkeytheorem{mythm}[name=Some Name]
\end{tcbwritetemp}

\begin{keythmscode}[withpreamble]
\begin{mythm}
Some text
\end{mythm}
\end{keythmscode}

\end{docKey}

\begin{docKey}{numbered}
  {=\docValue*{true}\textbar\docValue*{false}\textbar \docValue{unless-unique}}
  {default |true|, initially |true|}
For compatibility with \pkg{thmtools}, also accepts the values \docValue*{yes}, \docValue*{no}, and \docValue*{unless unique}.

\begin{tcbwritetemp}
\newkeytheorem{theorem*}[name=Theorem, numbered=false]
\end{tcbwritetemp}

\begin{keythmscode}[withpreamble]
\begin{theorem*}
An unnumbered theorem.
\end{theorem*}
\end{keythmscode}

\end{docKey}

\begin{docKey}{parent}
  {=\meta{counter}}
  {initially unset}
Aliases \docAuxKey{numberwithin} and \docAuxKey{within}.

\begin{tcbwritetemp}
\newkeytheorem{conjecture}[parent=section]
\end{tcbwritetemp}

\begin{keythmscode}[withpreamble]
\begin{conjecture}
The first number is the section.
\end{conjecture}
\end{keythmscode}

\end{docKey}

\begin{docKey}{sibling}{=\meta{counter}}{initially unset}
Aliases \docAuxKey{numberlike} and \docAuxKey{sharenumber}.

\begin{tcbwritetemp}
\newkeytheorem{lemma}[sibling=theorem]
\end{tcbwritetemp}

\begin{keythmscode}[withpreamble]
\begin{lemma}
This shares its counter with \texttt{theorem}.
\end{lemma}
\end{keythmscode}

\end{docKey}

\begin{docKey}{style}
  {=\meta{style name}}
  {initially unset}
Accepts any \meta{style name} defined by \refCom{newkeytheoremstyle}, as well as any of the predefined \pkg{amsthm} styles: \docValue{plain}, \docValue{definition}, and \docValue{remark}.

\begin{tcbwritetemp}
\newkeytheorem{remark}[style=remark]
\end{tcbwritetemp}

\begin{keythmscode}[withpreamble]
\begin{remark}
Some text
\end{remark}
\end{keythmscode}

\end{docKey}

\begin{docKeys}[
  doc parameter = {=\meta{code}},
  doc description = initially unset,
  ]
  {
    { doc name = preheadhook },
    { doc name = postheadhook },
    { doc name = prefoothook },
    { doc name = postfoothook },
  }
Details in \autoref{keythms-hooks}.

\begin{tcbwritetemp}
\newkeytheorem{test}[
  preheadhook=PREHEAD,
  postheadhook=POSTHEAD,
  prefoothook=PREFOOT,
  postfoothook=POSTFOOT
]
\end{tcbwritetemp}

\begin{keythmscode}[withpreamble]
\begin{test}
Some text
\end{test}
\end{keythmscode}

\end{docKeys}

\begin{docKey}{qed}
  {\colOpt{=\meta{symbol}}}
  {default \cs{qedsymbol}, initially unset}
Adds \colOpt{\meta{symbol}} to the end of the theorem body.
If no value is given, the symbol \qedsymbol{} is used.

\begin{tcbwritetemp}
\newkeytheorem{example}[qed]
\newkeytheorem{solution}[qed=$\clubsuit$]
\end{tcbwritetemp}

\begin{keythmscode}[withpreamble]
\begin{example}
Some text
\end{example}
\begin{solution}
Some more text
\end{solution}
\end{keythmscode}

\end{docKey}

\begin{docKey}{refname}
  {=\meta{ref name} \textrm{or} \brackets{\meta{singular name},\meta{plural name}}}
  {initially \meta{display name}}
If a single string, then the name used by \pkg{hyperref}'s \cs{autoref} and \pkg{cleveref}'s \cs{cref}.
If two strings separated by a comma, then the second string is the plural form used by \cs{cref}.
\end{docKey}

\begin{docKey}{Refname}
  {=\meta{ref name} \textrm{or} \brackets{\meta{singular name},\meta{plural name}}}
  {initially \meta{display name}}
Same as \refKey{refname} but for \docAuxCommand{Autoref} and \cs{Cref}.
Note that \cs{Autoref} is defined by \pkg{keytheorems}, but requires \pkg{hyperref} to work.
As with \cs{autoref}, there is also a starred version \docAuxCommand{Autoref*} that suppresses the hyperlink.

\begin{tcbwritetemp}
\newkeytheorem{prop}[
  name=Proposition,
  refname={proposition,propositions},
  Refname={Proposition,Propositions}
  ]
\end{tcbwritetemp}

\begin{keythmscode}[withpreamble]
\begin{prop}[label=abc]
Some text
\end{prop}
\begin{prop}[label=def]
Some more text
\end{prop}
\begin{theorem}
Consider \cref{abc,def}. \Autoref{abc} \dots
\end{theorem}
\end{keythmscode}

\end{docKey}

\subsection{Keys added by \pkg{keytheorems}} \label{thm-added-keys}

\begin{docKeys}[doc parameter={=\meta{length}}]
  {
    { doc name=leftmargin },
    { doc name=rightmargin },
  }
Sets the left (respectively, right) margin of the theorem relative to the text width.
This sets the theorem apart from the text, similar to a block quote. Code adapted from Enrico Gregorio's \TeX{} Stack Exchange answers:
\begin{itemize}
	\item \href{https://tex.stackexchange.com/a/67251/208544}{How to change margins in enunciation (theorem-like environment)?}
	\item \href{https://tex.stackexchange.com/a/236407/208544}{A theoremstyle with complete indentation using amsthm}
\end{itemize}
\begin{tcbwritetemp}
\newcommand{\marginthmtext}{%
  We need some text to show off theorems with margins. }
\newkeytheorem{quotethm}[
  name=Quote Theorem,
  leftmargin=1cm, rightmargin=1cm]
\newkeytheorem{indentedthm}[name=Indented Theorem, leftmargin=1cm]
\end{tcbwritetemp}

\newcommand{\marginthmtext}{We need some text to show off theorems with margins. }
\begin{keythmscode}[withpreamble]
\marginthmtext\marginthmtext\marginthmtext

\begin{quotethm}
\marginthmtext\marginthmtext\marginthmtext
\end{quotethm}

\marginthmtext\marginthmtext\marginthmtext

\begin{indentedthm}
\marginthmtext\marginthmtext\marginthmtext
\end{indentedthm}
\end{keythmscode}

\end{docKeys}

\begin{docKey}{tcolorbox}
  {\colOpt{=\marg{tcolorbox options}}}
  {initially unset}
This key specifies that the theorem be placed inside a tcolorbox environment with \colOpt{\meta{options}}.
The theorem head is typeset as a tcolorbox title; to avoid this see \refKey{tcolorbox-no-titlebar}.

\begin{tcbwritetemp}
\tcbset{
  defstyle/.style={
    arc=0mm,
    colback=blue!5!white,
    colframe=blue!75!black
    },
  }
\newkeytheorem{corollary}[tcolorbox]
\newkeytheorem{definition}[style=definition, tcolorbox={defstyle}]
\end{tcbwritetemp}

\begin{keythmscode}[withpreamble]
\begin{corollary}
Some text
\end{corollary}
\begin{definition}[A nice definition]
Some more text
\end{definition}
\end{keythmscode}

\end{docKey}

\begin{docKey}{tcolorbox-no-titlebar}
  {\colOpt{=\marg{tcolorbox options}}}
  {initially unset}
Same usage as \refKey{tcolorbox} but the theorem head is typeset as usual, not as a tcolorbox title.

\begin{tcbwritetemp}
\newkeytheorem{boxcor}[
  tcolorbox-no-titlebar={colback=red!10},
  name=Corollary, sibling=corollary
  ]
\end{tcbwritetemp}

\begin{keythmscode}[withpreamble]
\begin{boxcor}
Some text
\end{boxcor}
\end{keythmscode}

\end{docKey}

\pkg{tcolorbox} offers its own comprehensive \texttt{theorems} library. If all of your theorems are to be tcolorboxes, I highly recommend using it instead of this package! However, if only some of your theorems will use a tcolorbox, you may want to replicate the styles of \cs{NewTcbTheorem}. Here is an example that emulates \pkg{tcolorbox}'s \texttt{standard} theorem style.

\begin{tcbwritetemp}
\tcbset{
  thmstyle/.style={
    colback=green!5,
    colframe=green!35!black},
  }
\newkeytheoremstyle{tcb-standard}{
  tcolorbox=thmstyle,
  headpunct={},
  notebraces={}{},
  noteseparator={: },
  notefont=\bfseries,
  bodyfont=\normalfont,
  }
\newkeytheorem{mytheo}[
  name=My Theorem,
  style=tcb-standard]
\end{tcbwritetemp}

\begin{keythmscode}[withpreamble]
\begin{mytheo}[This is my title]
Some theorem text
\end{mytheo}
\end{keythmscode}

\section{Theorem styles}
\label{thmstyles}

\begin{docCommand}{newkeytheoremstyle}
  {\marg{name}\marg{options}}
This is \pkg{keytheorems}' version of \pkg{thmtools}' \cs{declaretheoremstyle}.
Since it makes little sense to define a style with no keys, we've made the \meta{options} argument mandatory.
Note that unlike \pkg{amsthm}'s \cs{newtheoremstyle}, this command will error if a style has already been defined.
To overwrite an existing style, there is the analogous \docAuxCommand{renewkeytheoremstyle}.
For completeness, there are also \docAuxCommand{providekeytheoremstyle} and \docAuxCommand{declarekeytheoremstyle}.

The defined style can be used with either the \refKey{style} key or the traditional \cs{theoremstyle}.
\end{docCommand}

\begin{notebox}
For the AMS classes \cls{amsart}, \cls{amsbook}, and \cls{amsproc}, as well as the \cls{amsart}-based \cls{acmart} and \cls{aomart}, the initial key values are slightly different than what's below in order to match those class's defaults. See \autoref{class-support} for details.
\end{notebox}

\subsection{Keys also defined in \pkg{thmtools}}

The following keys have the same meaning and syntax as the corresponding \pkg{thmtools} keys.
In addition to the list below, most of the keys available to \refCom{newkeytheorem} can be used in \cs{newkeytheoremstyle}.

\begin{docKey}{bodyfont}
  {=\meta{font declarations}}
  {initially \cs{itshape}}

\end{docKey}

\begin{docKey}{break}
  {}
  {initially unset}
Do not use this with the \refKey{postheadspace} key.
\end{docKey}

\begin{docKey}{headfont}
  {=\meta{font declarations}}
  {initially \cs{bfseries}}

\end{docKey}

\begin{docKey}{headformat}
  {=\docValue{margin}\textbar\docValue{swapnumber}\textbar\meta{code using \cs{NAME}, \cs{NUMBER}, and \cs{NOTE}\,}}
  {}
Alias \docAuxKey{headstyle}. Within \meta{code}, the commands \docAuxCommand{NAME}, \docAuxCommand{NUMBER}, and \docAuxCommand{NOTE} correspond to the formatted parts of the theorem head.
\end{docKey}

\begin{notebox}
In \refKey{headformat}, you may also use the traditional \pkg{amsthm} commands \cs{thmname}, \cs{thmnumber}, and \cs{thmnote}, where |#1| is the theorem name, |#2| the number, and |#3| the note. \pkg{keytheorems} expands the head spec inside |\text_expand:n| so for these commands to work properly, the package adds them to |\l_text_expand_exclude_tl|. Note also that if you use these lower-level commands, the style keys \refKey{notebraces}, \refKey{notefont}, \refKey{noteseparator}, and \refKey{numberfont} will have no effect (of course, you can manually control these things inside the commands' arguments).
\end{notebox}

\begin{docKey}{headindent}
  {=\meta{length}}
  {initially |0pt|}

\end{docKey}

\begin{docKey}{headpunct}
  {=\meta{code}}
  {initially \ttbraces{.}}

\end{docKey}

\begin{docKey}{notebraces}
  {=\marg{left brace}\marg{right brace}}
  {initially \ttbraces{(}\ttbraces{)}}

\end{docKey}

\begin{docKey}{notefont}
  {=\meta{font declarations}}
  {initially \cs{fontseries}\cs{mddefault}\cs{upshape}}

\end{docKey}

\begin{docKey}{postheadspace}
  {=\meta{length}}
  {initially |5pt plus 1pt minus 1pt|}
Do not use this with the \refKey{break} key.
\end{docKey}

\begin{docKey}{spaceabove}
  {=\meta{length}}
  {initially \cs{topsep}}

\end{docKey}

\begin{docKey}{spacebelow}
  {=\meta{length}}
  {initially \cs{topsep}}

\end{docKey}

\begin{notebox}
With \refKey{tcolorbox} and \refKey{tcolorbox-no-titlebar}, the \refKey{spaceabove} and \refKey{spacebelow} keys are internally passed to \pkg{tcolorbox}'s \texttt{before skip} and \texttt{after skip}. When no explicit \texttt{spaceabove} or \texttt{spacebelow} values are given, \pkg{tcolorbox} defaults are used instead of \cs{topsep}.
\end{notebox}

\subsection{Keys added by \pkg{keytheorems}}

\begin{docKey}{inherit-style}
  {=\meta{style name}}
  {initially unset}
Inherit the keys of any style declared with \refCom{newkeytheoremstyle}.
Additionally, the three styles predefined by \pkg{amsthm} are possible values: \docValue{plain}, \docValue{definition}, and \docValue{remark}.
\end{docKey}

\begin{docKey}{noteseparator}
  {=\meta{code}}
  {initially \textvisiblespace}
The code inserted before the note, and printed only if there is a note. This is executed \emph{before} the font commands set by \refKey{notefont} take effect.
\end{docKey}

\begin{docKey}{numberfont}
  {=\meta{font declarations}}
  {initially |\upshape|}
For almost all theorem styles, it is recommended that you \emph{do not} change this setting.
\end{docKey}

\section{Restating theorems}

When a theorem is given the \refKey{store} key, the contents of the theorem are saved and written to a |.thlist| file.
At the start of the next run, this file is input at the beginning of the document and allows you to retrieve the stored theorems at any point, before or after the original theorem.

\begin{docCommand}{getkeytheorem}
  {\oarg{property}\marg{tag}}
Retrieves the theorem given the key |store=|\meta{tag} or |store*=|\meta{tag}.
An optional \meta{property} can be given to retrieve only the corresponding part of the theorem.
Currently only the property |body| is implemented, which retrieves the (unformatted) body of the theorem.

\begin{keythmscode}[]
\getkeytheorem{mytag}

\begin{example}[store=mytag]
Fascinating example.
\end{example}

\getkeytheorem[body]{mytag}
\end{keythmscode}

\end{docCommand}

\begin{docCommands}
  {
    {
      doc name = IfRestatingTF,
      doc parameter=\marg{true code}\marg{false code}
    },
    {
      doc name = IfRestatingT,
      doc parameter=\marg{true code}
    },
    {
      doc name = IfRestatingF,
      doc parameter=\marg{false code}
    }
  }
Executes \meta{true code} if being retrieved with \refCom{getkeytheorem} and \meta{false code} if in the original theorem. This is reversed if \texttt{store*} is used.

\begin{keythmscode}[]
\begin{example}[store=hmm]
I am the
\IfRestatingTF{restated}{original}
example!
\end{example}

\getkeytheorem{hmm}
\end{keythmscode}

\end{docCommands}

\section{Listing theorems}
\label{listingthms}

\begin{docCommand}{listofkeytheorems}
  {\oarg{options}}
Similar to \cs{listoffigures} or \cs{listoftables} but for theorems.
For \cls{memoir} and the AMS classes, \pkg{keytheorems} tries to copy the formatting of these commands as defined by the class.
For other classes, manual adjustments to \refKey{numwidth} and \refKey{indent} may be necessary.
\end{docCommand}

\begin{docCommand}{keytheoremlistset}
  {\marg{options}}

\end{docCommand}

\begin{keythmscode}[]
\listofkeytheorems
\end{keythmscode}

\subsection{Keys also defined in \pkg{thmtools}}

\begin{docKey}{ignore}
  {=\marg{comma-list of env names}}
  {initially unset}

\end{docKey}

\begin{docKey}{ignoreall}
  {}
  {initially unset}
  
\begin{keythmscode}[]
\listofkeytheorems[ignoreall,show=theorem]
\listofkeytheorems[
  ignoreall, show=conjecture,
  title=List of Conjectures
  ]
\end{keythmscode}

\end{docKey}

\begin{docKey}{numwidth}
  {=\meta{length}}
  {initially |2.3em|}
For the AMS classes, this is initially |1.5pc|.
\end{docKey}

\begin{docKey}{onlynamed}
  {\colOpt{=\marg{comma-list of env names}}}
  {initially unset}

\end{docKey}

\begin{docKey}{show}
  {=\marg{comma-list of env names}}
  {initially all theorems}

\end{docKey}

\begin{docKey}{showall}
  {}
  {initially set}

\end{docKey}

\begin{docKey}{swapnumber}
  {\colOpt{=true\textbar false}}
  {initially |false|}

\end{docKey}

\begin{docKey}{title}
  {=\meta{text}}
  {initially \cs{GetTranslation}\ttbraces{keythms\string_listof\string_title}}
Defaults to ``List of Theorems'' if English or an unknown language is used.
Currently French, German, Italian, Portuguese, and Spanish have (likely inaccurate!) translations.
A translation can be added with a GitHub pull request or manually with
\begin{center}
\cs{DeclareTranslation}\marg{lang}\ttbraces{keythms\string_listof\string_title}\marg{text}.
\end{center}
\end{docKey}

\subsection{Keys added by \pkg{keytheorems}}

\begin{docKey}{format-code}
  {=\meta{code with \textup{\texttt{\#1}}, \textup{\texttt{\#2}}, and \textup{\texttt{\#3}}}}
  {initially \cs{numberline}\texttt{\{\#2\}\#1\#3}}
Allows full control over the format for list entries. The theorem name is |#1|, the number is |#2|, and the (formatted) note is |#3|. The note formatting is still controlled by \refKey{note-code}.
\end{docKey}

\begin{docKey}{indent}
  {=\meta{length}}
  {initially |1.5em|}
Sets the left indent of items in the list of theorems.
For \cls{memoir} and the AMS classes, the indent is initially |0pt|.
It is not recommended to change this unless your class has different defaults not already covered.
\end{docKey}

\begin{docKey}{no-chapter-skip}
  {\colOpt{=true\textbar false}}
  {initially |false|}
By default a small vertical space is inserted between each chapter's chunk of theorems.
Setting this key to |true| removes this space.
\end{docKey}

\begin{docKey}{chapter-skip-length}
  {=\meta{length}}
  {initially |10pt|}
Controls the amount of space inserted between chunks.
\end{docKey}

\begin{docKey}{no-continues}
  {\colOpt{=true\textbar false}}
  {initially |false|}
Suppresses the printing of theorems given the \refKey{continues} key in the list of theorems.
\end{docKey}

\begin{docKey}{no-title}
  {\colOpt{=true\textbar false}}
  {initially |false|}
Suppresses the title of the list of theorems.
Useful for custom ordering of the list.

\begin{keythmscode}[]
\keytheoremlistset{ignoreall}
\listofkeytheorems[show=example]
\listofkeytheorems[show=solution, no-title]
\end{keythmscode}

\end{docKey}

\begin{docKey}{no-toc}
  {\colOpt{=true\textbar false}}
  {initially false}
With the standard classes, lists of figures/tables are not added to the table of contents by default. The same is true for \cs{listofkeytheorems}, and with those classes this key does nothing. However some classes, notably \cls{memoir} and the AMS classes, do add lists to the table of contents. With these classes, this key suppresses the addition of the list of theorems to the table of contents.
\end{docKey}

\begin{docKey}{note-code}
  {=\meta{code with \textup{\texttt{\#1}}}}
  {initially \ttbraces{ (\#1)}}
Formats the optional note in the list of theorems.
\end{docKey}

\begin{docKey}{onlynumbered}
  {\colOpt{=\marg{comma-list of env names}}}
  {initially unset}
Similar to \refKey{onlynamed}, but lists only those theorems which are numbered.
This is useful if you'd like to exclude things like unnumbered definitions and remarks from the list of theorems.
\end{docKey}

\begin{docKey}{print-body}
  {}
  {initially unset}
Instead of listing the theorem headings, the theorems are restated with their body text.
Not very useful without the \refKey{store-all} load-time option.
\end{docKey}

\begin{docKey}[][doc label=listof/seq]{seq}
  {=\meta{name}}
  {initially unset}
Used to list only the theorems added to the custom sequence \meta{name} with the \refKey{thm/seq} theorem key.
This is the only way to fully customize which theorems appear in the list of theorems.
Unlike with \refKey{show}, you do not need to use \refKey{ignoreall} to prevent theorems not in \meta{name} from being printed.
\end{docKey}

\begin{docKey}{title-code}
  {=\meta{code with \textup{\texttt{\#1}}}}
  {initially \cs{section*}\ttbraces{\#1}}
If \cs{chapter} is defined, then initially this is instead \cs{chapter*}\ttbraces{\#1}. This key has no effect if used with an AMS class because these classes hard-code the section heading into \cs{@starttoc}.
\end{docKey}

\subsection{Adding code to list of theorems}

There are analogous commands to \cs{addcontentsline} and \cs{addtocontents} for adding entries or arbitrary code to the list of theorems.

\begin{notebox}
You \emph{must} use these commands rather than the aforementioned because the |.thlist| file is also used to define restated theorems and cannot contain unexpected code.
\end{notebox}

\begin{docCommand}{addtheoremcontentsline}
  {\marg{level}\marg{text}}

\end{docCommand}

\begin{docCommand}{addtotheoremcontents}
  {\marg{code}}

\end{docCommand}

\section{Theorem hooks} \label{keythms-hooks}

\begin{docCommand}{addtotheoremhook}
  {\oarg{env name}\marg{hook name}\marg{code}}
The \meta{hook name} can be \hook{prehead}, \hook{posthead}, \hook{prefoot}, \hook{postfoot}, or \hook{restated}.
If no \meta{env name} is given, the \meta{code} is added to the ``generic'' hook, i.e. applied to all theorems.
As in \pkg{thmtools}, the order of hooks is as follows:

\begin{center}
  \small
  \tcbset{
    nobeforeafter,
    tikznode,
    tcbox width=forced center,
    width=1.85cm,
    tcbox raise=-0.4cm,
    size=fbox,
    after=\;$\rightarrow$
    }
  \tcbox{\meta{env name}\\ \hook{prehead}}
  \tcbox{generic\\ \hook{prehead}}
  \tcbox[width=3.25cm,tcbox raise=-0.2cm]{\cs{begin}\brackets{\meta{env name}}}
  \tcbox{\meta{env name}\\ \hook{posthead}}
  \tcbox[after={}]{generic\\ \hook{posthead}}
  \\[2ex]
  \tcbset{before=$\rightarrow$\;}
  \tcbox[width=2.25cm,tcbox raise=-0.2cm]{\meta{theorem body}}
  \\[2ex]
  \tcbset{after={}}
  \tcbox[before={}]{generic\\ \hook{prefoot}}
  \tcbox{\meta{env name}\\ \hook{prefoot}}
  \tcbox[width=3.25cm,tcbox raise=-0.2cm]{\cs{end}\brackets{\meta{env name}}}
  \tcbox{generic\\ \hook{postfoot}}
  \tcbox{\meta{env name}\\ \hook{postfoot}}
\end{center}
The \hook{restated} hook is applied at the start of theorems retrieved with the command \cs{getkeytheorem}, after the \hook{prehead} hook. 
This can be useful for disabling commands such as \cs{footnote} in the restated theorems, e.g.
\begin{dispListing}
\addtotheoremhook{restated}{\renewcommand\footnote[2][]{}}
\end{dispListing}

By default, \pkg{keytheorems} disables the \cs{label} and \cs{RecordProperties} commands in restated theorems.

In \pkg{thmtools}, the \hook{prefoot} and \hook{postfoot} hooks always prepend code, i.e. the code
\begin{dispListing}
\addtotheorempostfoothook{A}
\addtotheorempostfoothook{B}
\end{dispListing}
results in |BA| after the theorem.
With \pkg{keytheorems}, code is added in the order declared, meaning
\begin{dispListing}
\addtotheoremhook{postfoot}{A}
\addtotheoremhook{postfoot}{B}
\end{dispListing}
results in |AB| after the theorem.
This is the behavior of the \LaTeX{} kernel hooks that \pkg{keytheorems} uses under the hood.

Code added using the hook keys \refKey{preheadhook}, etc. is outermost, meaning executed first in \hook{prehead} and \hook{posthead} and last in \hook{prefoot} and \hook{postfoot}.
\end{docCommand}

\section{Miscellaneous notes}

\subsection{\cls{beamer} support}

The package contains some \emph{highly experimental} code to support theorems with \cls{beamer}, including overlays.
Most style keys are disabled by the |default| \cls{beamer} theorem template.
More become functional by setting
\begin{dispListing}
\setbeamertemplate{theorems}[ams style]
\end{dispListing}
in the preamble.
Alternatively, you have full control of theorems by setting the class option |noamsthm|.

Note that by default \cls{beamer} defines a set of theorems when the class is loaded.
These can be overwritten with \refCom{renewkeytheorem} or disabled entirely with the |notheorems| class option.

Due to complications with overlays, writing contents of theorems to the |thlist| file is disabled. This means theorems can only be restated \emph{after} their original statement. Furthermore, \refCom{listofkeytheorems} is disabled and a warning issued if used.

User feedback is necessary to make this code fully compatible. Please report issues on the \href{https://github.com/mbertucci47/keytheorems/issues}{Github page}!

\subsection{Support for other classes}
\label{class-support}

As mentioned in \autoref{thmstyles}, the initial style key values set by \pkg{keytheorems} are adjusted for the AMS classes \cls{amsart}, \cls{amsbook}, and \cls{amsproc}, the \cls{amsart}-based \cls{acmart} and \cls{aomart}, and \cls{jlreq}. You can find the exact changed values in the support files \texttt{keythms-\meta{class}-support.tex}.

These class support files also contain code to adapt to class' formatting of lists-of as mentioned in \autoref{listingthms}; changes are made for the AMS classes, \cls{memoir}, \cls{IEEEtran}, and \cls{jlreq}.

\subsection{Support for font packages}

Some font packages, all by Michael Sharpe, offer a |theoremfont| option that redefines the |plain| style body font to have italic text with upright figures, punctuation, and delimiters. \pkg{keytheorems} detects this option and sets its initial style values accordingly. The supported packages are \pkg{baskervillef}, \pkg{cochineal}, \pkg{libertinust1math}, \pkg{newpxtext}, \pkg{newtxtext}, \pkg{scholax}, and \pkg{XCharter}.

\subsection{Support for tagged PDF}

The \LaTeX{} team has been working hard to support the creation of tagged PDFs (see \url{https://latex3.github.io/tagging-project/}).
The current |dev| formats make \pkg{amsthm} compatible with the kernel tagging code.
Most of \pkg{keytheorems} is supported too, and anything that doesn't work should be reported.
Explicitly not supported are the \refKey{tcolorbox} and \refKey{tcolorbox-no-titlebar} keys.
There is only partial support for \refKey{leftmargin} and \refKey{rightmargin}, namely only the latter works.

To produce a tagged PDF, add \cs{DocumentMetadata} in the first line of your document (additional instructions are found on the Tagging Project \href{https://latex3.github.io/tagging-project/documentation/prototype-usage-instructions.html}{website}).
An example invocation might look like

\begin{dispListing}
\DocumentMetadata
  {
    lang=en-US,
    pdfversion=2.0,
    pdfstandard=ua-2,
    testphase={phase-III,math,table,title,firstaid}
  }
\end{dispListing}

At a minimum, the |testphase| modules |phase-III| and |firstaid| are required.
The GitHub tests folder contains an example of a tagged PDF using \pkg{keytheorems}: \href{https://github.com/mbertucci47/keytheorems/blob/main/tests/tagged-keytheorems-amsthmtest.tex}{\texttt{tagged-keytheorems-amsthmtest.tex}}.

\subsection{Public coding interfaces}

\begin{docCommand}{l_keythms_thmuse_envname_tl}{}
Inside theorem environments and in all theorem hooks, you have access to the theorem's environment and counter name in this token list variable.
\end{docCommand}

\begin{docKeys}[doc no index,color key=black]
  {
    {
      doc name=keytheorems/allthms/\textmd{\meta{hook name}},
      doc label=allthms-hook,
    },
    {
      doc name=keytheorems/\textmd{\meta{envname}}/\textmd{\meta{hook name}},
      doc label=envname-hook,
    }
  }
These are the ``real'' names for the hooks described in \autoref{keythms-hooks}.
They can be useful with \cs{AddToHookNext} or the kernel's label mechanism for hooks.
\end{docKeys}

\section{Further examples}

More examples will be added soon -- rather, eventually\dots{}
For now, you can find a \pkg{keytheorems} adaptation of \pkg{amsthm}'s classic file \href{https://mirrors.ctan.org/macros/latex/required/amscls/doc/thmtest.tex}{\texttt{thmtest.tex}} in the Github \texttt{tests} folder: \href{https://github.com/mbertucci47/keytheorems/blob/main/tests/keytheorems-amsthmtest.tex}{\texttt{keytheorems-amsthmtest.tex}}.

%\section{Implementation}
%\inputminted[linenos,style=bw]{latex}{../keytheorems.sty}

\printindex

\end{document}