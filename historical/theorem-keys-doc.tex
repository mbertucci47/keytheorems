% !TeX program = pdflatex --synctex=1 --interaction=nonstopmode %.tex | makeindex %.idx | pdflatex --synctex=1 --interaction=nonstopmode %.tex | txs:///view-pdf
\PassOptionsToPackage{hyperindex}{hyperref}
\PassOptionsToPackage{dvipsnames}{xcolor}
\documentclass{ltxdoc}
\usepackage{geometry}
\geometry{margin=1in}
\usepackage{tcolorbox}
\tcbuselibrary{documentation}
\usepackage{theorem-keys}

\def\version{0.0.1$\alpha$}

\title{
	\pkg{theorem-keys} package \\[1ex]
	\large version \version \\[1ex]
	\href{https://github.com/mbertucci47/theorem-keys}
	  {\texttt{github.com/mbertucci47/theorem-keys}}
	}
\author{Matthew Bertucci}

\NewThm{mytheorem}[name=Some Name]

\MakeShortVerb{\|}
\newcommand{\env}{\texttt}
\newcommand{\hook}{\texttt}
\NewCommandCopy{\braces}{\brackets}
\newcommand{\bracks}[1]{\texttt{[#1]}}
\newcommand{\ttbraces}[1]{\braces{\texttt{#1}}}
%\tcbset{index gather commands=false}
\tcbset{
	color command = BrickRed,
	color key = Fuchsia,
	color value = teal,
	}
\tcbset{
  preamble/.style = {
    before upper = {%
      {\ttfamily\itshape \% preamble}%
      \tcbusetemplisting%
      \medskip
      {\ttfamily\itshape \% document}%
      }
    }
  }

\NewTCBListing{thmkeyscode}{ O{} }{
	colback=red!5!white,
	colframe=red!75!black,
	listing side text,
	#1
	}

\makeindex

\begin{document}

\maketitle

\begin{abstract}
An \pkg{expl3}-implementation of a key-value interface to \pkg{amsthm}, implementing most of the functionality provided by \pkg{thmtools}. Full of bugs and incomplete. Don't use it for anything important!
\end{abstract}

\begin{tcolorbox}[colframe=red,colback=red!10!white]
\textbf{Warning!} In addition to being a completely unreliable package, the documentation is just a skeleton! For full details look at the package code.
\end{tcolorbox}

\section{Load-time Options}

\begin{docKey}{overload}{}{initially unset}
Redefines \cs{newtheorem} to internally use the \pkg{theorem-keys} machinery. The syntax remains the same.
\end{docKey}

\begin{docKey}{thmtools-compat}{}{initially unset}
For compatibility with \pkg{thmtools} syntax. Currently defines the \docAuxCommand{declaretheoremstyle}, \docAuxCommand{declaretheorem}, and \docAuxCommand{listoftheorems} commands and the \docAuxEnvironment{restatable} environment. 
\end{docKey}

\begin{docKey}{store-all}{}{initially unset}
Tells \pkg{theorem-keys} to grab the body of each theorem so it can later be
printed with |\ListOfThms[print-body]|. Note that this means a theorem body
\emph{cannot} contain verbatim material.
\end{docKey}
    
\section{Global Options}

\begin{docCommand}{ThmKeysSet}{\marg{options}}

\end{docCommand}

\begin{docKey}{restate-counters}{=\marg{comma-list of counters}}{initially \ttbraces{equation}}
The counters whose values are preserved when a theorem is restated.
\end{docKey}

\begin{docKey}{continues-code}{=\meta{code with \textup{\texttt{\#1}}}}{initially |continuing from p.\cs{,}\cs{pageref}\ttbraces{\#1}|}

\end{docKey}

\begin{docKey}{qed-symbol}{=\meta{symbol}}{initially \cs{openbox}}
Redefines \cs{qedsymbol} to be \meta{symbol}.
\end{docKey}

\section{Theorem Styles}

\begin{docCommand}{NewThmStyle}{\marg{name}\marg{options}}
This is \pkg{theorem-keys}' version of \pkg{thmtools}' \cs{declaretheoremstyle}\bracks{\meta{options}}\marg{name}. Since it makes little sense to define a style with no keys, we've made the \meta{options} argument mandatory.
\end{docCommand}

\subsection{Keys inherited from \pkg{thmtools}}

The following keys have the same meaning and syntax as the corresponding \pkg{thmtools} keys.

\begin{docKey}{spaceabove}{=\meta{length}}{initially \cs{topsep}}

\end{docKey}

\begin{docKey}{spacebelow}{=\meta{length}}{initially \cs{topsep}}

\end{docKey}

\begin{docKey}{bodyfont}{=\meta{font declarations}}{initially \cs{itshape}}

\end{docKey}

\begin{docKey}{headindent}{=\meta{length}}{initially |0pt|}

\end{docKey}

\begin{docKey}{headfont}{=\meta{font declarations}}{initially \cs{bfseries}}

\end{docKey}

\begin{docKey}{headpunct}{=\meta{code}}{initially \ttbraces{.}}

\end{docKey}

\begin{docKey}{postheadspace}{=\meta{length}}{initially |5pt plus 1pt minus 1pt|}
Do not use this with the \refKey{break} key.
\end{docKey}

\begin{docKey}{break}{}{initially unset}
Do not use this with the \refKey{postheadspace} key.
\end{docKey}

\begin{docKey}{notefont}{=\meta{font declarations}}{initially \cs{fontseries}\cs{mddefault}\cs{upshape}}

\end{docKey}

\begin{docKey}{notebraces}{=\marg{left brace}\marg{right brace}}{initially \ttbraces{(}\ttbraces{)}}

\end{docKey}

\begin{docKey}{headstyle}{=\docValue{margin}\textbar\docValue{swapnumber}\textbar\meta{code using \cs{NAME}, \cs{NUMBER}, and \cs{NOTE}}}{}
Alias \docAuxKey{headstyle}. Within \meta{code}, the commands \docAuxCommand{NAME}, \docAuxCommand{NUMBER}, and \docAuxCommand{NOTE} correspond to the formatted parts of the theorem head.
\end{docKey}

\subsection{Keys added by \pkg{theorem-keys}}

\begin{docKey}{inherit-style}{=\meta{style name}}{initially unset}
Inherit the keys of any style declared with \refCom{NewThmStyle}. Additionally, the three styles predefined by \pkg{amsthm} are possible values: \docValue{plain}, \docValue{definition}, and \docValue{remark}.
\end{docKey}

\section{Declaring Theorems}

\begin{docCommand}{NewThm}{\marg{env name}\oarg{options}}
Defines a theorem environment \meta{env name} which itself takes a few options (see \autoref{in-doc-keys}). You can also declare multiple theorems at once by replacing \meta{env name} with a comma-list of names, e.g. \cs{NewThm}\ttbraces{theorem,lemma,proposition}\bracks{\meta{options}}
\end{docCommand}

\begin{tcbwritetemp}
\NewThm{mytheorem}[name=Some Name]
\end{tcbwritetemp}

\begin{thmkeyscode}[preamble]
\begin{mytheorem}[a note]
Some text
\end{mytheorem}

\begin{mytheorem}[name=heading,label=foo]
Some more text
\end{mytheorem}

\ref{foo}
\end{thmkeyscode}

\subsection{Keys inherited from \pkg{thmtools}}

\begin{docKey}[][doc label=thm/name]{name}{=\meta{display name}}{initially title-cased \meta{env name}}
Aliases \docAuxKey{title} and \docAuxKey{heading}.
\end{docKey}

\begin{docKey}{numbered}{\colOpt{=true\textbar false}}{default |true|, initially |true|}
For compatibility with \pkg{thmtools}, also accepts the values \colOpt{|yes|} and \colOpt{|no|}.
\end{docKey}

\begin{docKey}{parent}{=\meta{counter}}{initially unset}
Aliases \docAuxKey{numberwithin} and \docAuxKey{within}.
\end{docKey}

\begin{docKey}{sibling}{=\meta{counter}}{initially unset}
Aliases \docAuxKey{numberlike} and \docAuxKey{sharenumber}.
\end{docKey}

\begin{docKey}{style}{=\meta{style name}}{initially unset}

\end{docKey}

\begin{docKeys}[
  doc parameter = {=\meta{code}},
  doc description = initially unset,
  ]
  {
	{ doc name = preheadhook },
	{ doc name = postheadhook },
	{ doc name = prefoothook },
	{ doc name = postfoothook },
  }
Details in \autoref{thmkeys-hooks}.
\end{docKeys}

\begin{docKey}{refname}{=\meta{ref name} or \brackets{\meta{singular name},\meta{plural name}}}{initially \meta{display name}}
If a single string, then the name used by \pkg{hyperref}'s \cs{autoref} and \pkg{cleveref}'s \cs{cref}. If two strings separated by a comma, then the second string is the plural form used by \cs{cref}.
\end{docKey}

\begin{docKey}{Refname}{=\meta{ref name} or \brackets{\meta{singular name},\meta{plural name}}}{initially \meta{display name}}
Same as \refKey{refname} but for \cs{Autoref} (not yet implemented!) and \cs{Cref}.
\end{docKey}

\begin{docKey}{qed}{\colOpt{=\meta{symbol}}}{default \cs{openbox}, initially unset}

\end{docKey}

\subsection{Keys added by \pkg{theorem-keys}}

\begin{docKey}{tcolorbox}{\colOpt{=\marg{tcolorbox options}}}{initially unset}

\end{docKey}

\section{Keys available to theorem environments} \label{in-doc-keys}

\begin{docKey}{label}{=\meta{label name}}{initially unset}

\end{docKey}

\begin{docKey}{note}{=\meta{text}}{initially unset}
Alias \docAuxKey[][doc label=thmuse/name]{name}.
\end{docKey}

\begin{docKey}{continues}{=\meta{label name}}{initially unset}

\end{docKey}

\begin{docKey}{store}{=\meta{csname}}{initially unset}
Alias \docAuxKey{restate}. Defines a command \texttt{\char`\\}\meta{csname} that can be used to restate the theorem, including the body text, later in the document.
\end{docKey}

\section{Listing Theorems}

\begin{docCommand}{ListOfThms}{\oarg{options}}

\end{docCommand}

\begin{docCommand}{ThmKeysListSet}{\marg{options}}

\end{docCommand}

\subsection{Keys inherited from \pkg{thmtools}}

\begin{docKey}{numwidth}{=\meta{length}}{initially |2.3em|}

\end{docKey}

\begin{docKey}{ignore}{=\marg{comma-list of env names}}{initially unset}

\end{docKey}

\begin{docKey}{show}{=\marg{comma-list of env names}}{initially all theorems}

\end{docKey}

\begin{docKey}{onlynamed}{\colOpt{=\marg{comma-list of env names}}}{initially unset}

\end{docKey}

\begin{docKey}{ignoreall}{}{initially unset}

\end{docKey}

\begin{docKey}{showall}{}{initially set}

\end{docKey}

\begin{docKey}{title}{=\meta{text}}{initially |List of Theorems|}

\end{docKey}

\begin{docKey}{swapnumber}{\colOpt{=true\textbar false}}{initially |false|}

\end{docKey}

\subsection{Keys added by \pkg{theorem-keys}}

\begin{docKey}{title-code}{=\meta{code with \textup{\texttt{\#1}}}}{initially \cs{section*}\ttbraces{\#1}}
If \cs{chapter} is defined, then initially this is instead \cs{chapter*}\ttbraces{\#1}.
\end{docKey}

\begin{docKey}{no-title}{}{initially unset}

\end{docKey}

\begin{docKey}{print-body}{}{initially unset}
Instead of listing the theorem headings, the theorems are restated with their body text.
\end{docKey}

\section{Theorem Hooks} \label{thmkeys-hooks}

\begin{docCommands}[doc parameter = \oarg{env name}\marg{code}]
  {
	{ doc name = AddToTheoremPreHeadHook },
	{ doc name = AddToTheoremPostHeadHook },
	{ doc name = AddToTheoremPreFootHook },
	{ doc name = AddToTheoremPostFootHook },
  }
As in \pkg{thmtools}, the order of hooks is as follows:
\begin{center}
  \tcbset{
    nobeforeafter,
    tikznode,
    tcbox width=forced center,
    width=2cm,
    tcbox raise=-0.4cm,
    size=fbox,
    after=\;$\Rightarrow$
    }
  \tcbox{\meta{env name}\\ \hook{prehead}}
  \tcbox{generic\\ \hook{prehead}}
  \tcbox[width=3.5cm,tcbox raise=-0.2cm]{\cs{begin}\brackets{\meta{env name}}}
  \tcbox{\meta{env name}\\ \hook{posthead}}
  \tcbox[after={}]{generic\\ \hook{posthead}}
  \\[2ex]
  \tcbset{before=$\Rightarrow$\;}
  \tcbox[width=2.5cm,tcbox raise=-0.2cm]{\meta{theorem body}}
  \\[2ex]
  \tcbset{after={}}
  \tcbox[before={}]{generic\\ \hook{prefoot}}
  \tcbox{\meta{env name}\\ \hook{prefoot}}
  \tcbox[width=3.5cm,tcbox raise=-0.2cm]{\cs{end}\brackets{\meta{env name}}}
  \tcbox{generic\\ \hook{postfoot}}
  \tcbox{\meta{env name}\\ \hook{postfoot}}
\end{center}
In \pkg{thmtools}, the \hook{prefoot} and \hook{postfoot} hooks always prepend code, i.e. the code
\begin{dispListing}
\addtotheorempostfoothook{A}
\addtotheorempostfoothook{B}
\end{dispListing}
results in |BA| after the theorem. With \pkg{theorem-keys}, code is added in the order declared, meaning
\begin{dispListing}
\AddToTheoremPostFootHook{A}
\AddToTheoremPostFootHook{B}
\end{dispListing}
results in |AB| after the theorem. This is the behavior of the \LaTeX{} kernel hooks that \pkg{theorem-keys} uses under the hood.

Right now, code added using the hook keys \refKey{preheadhook}, etc. is outermost, meaning executed first in \hook{prehead} and \hook{posthead} and last in \hook{prefoot} and \hook{postfoot}. This may change if I think of good reasons to do so\dots
\end{docCommands}

\printindex

\end{document}