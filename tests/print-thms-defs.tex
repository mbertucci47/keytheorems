\ifthmtools
\usepackage{amsthm,thmtools}
\newcommand{\newkeytheoremstyle}[2]{\declaretheoremstyle[#2]{#1}}
\NewDocumentCommand{\newkeytheorem}{mO{}}{\declaretheorem[#2]{#1}}
\newcommand{\listofkeytheorems}{\listoftheorems}
\else
\usepackage{keytheorems}
\keytheoremset{thmtools-compat}
\fi

\usepackage{kantlipsum}
\usepackage[dvipsnames]{xcolor}
\usepackage{hyperref,cleveref}

\newkeytheorem{theorem}

\newkeytheorem{theoremS}[
	numberwithin=section
	]
\newkeytheorem{exercise}[
	name=\"Ubung
	]
\newkeytheorem{lem}[
	name=Lemma,
	sibling=theorem,
	]
\newkeytheorem{euclid}[
	numbered=no,
	name=Euclid's Prime Theorem
	]
\newkeytheorem{singleton}[
	numbered=unless unique
	]
\newkeytheorem{couple}[
	numbered=unless unique
	]
\newkeytheorem{remark}[
	style=remark
	]
\newkeytheorem{BoxI}[
    shaded={bgcolor=Lavender,textwidth=12em}
    ]
\newkeytheorem{BoxII}[
    shaded={rulecolor=Lavender,rulewidth=2pt,bgcolor=white}
    ]
\newkeytheorem{boxtheorem L}[
    thmbox=L
    ]
\newkeytheorem{boxtheorem M}[
    thmbox=M
    ]
\newkeytheorem{boxtheorem S}[
    thmbox=S
    ]
\newkeytheoremstyle{mystyle}{
	spaceabove=6pt, spacebelow=6pt,
	headfont=\normalfont\bfseries,
	notefont=\mdseries, notebraces={(}{)},
	bodyfont=\normalfont,
	postheadspace=1em,
	}
\newkeytheorem{styledtheorem}[
	style=mystyle,
	qed=\qedsymbol,
	]
\newkeytheorem{callmeal}[
	name=Theorem,
	refname={theorem,theorems},
    Refname={Theorem,Theorems}
    ]
\newkeytheorem{noname}[
	name={}
	]
\newkeytheorem{nonamenonumber}[
	name={},
	numbered=no
	]

\ExplSyntaxOn
\int_new:N \l_my_style_int
\clist_new:N \l_my_style_clist
\clist_set:Nn \l_my_style_clist {
	spaceabove=20pt,
	spacebelow=20pt,
	bodyfont=\tiny,
	headindent=20pt,
	headfont=\tiny,
	headpunct={:},
	postheadspace=20pt,
	break,
	notefont=\tiny,
	notebraces={[}{]},
	headstyle=margin,
	headstyle=swapnumber,
	headstyle={\NAME\NOTE\ \NUMBER},
	}
\clist_map_inline:Nn \l_my_style_clist {
	\int_incr:N \l_my_style_int
	\exp_args:Ne \newkeytheoremstyle { mysty \int_use:N \l_my_style_int } { #1 }
	\exp_args:Ne \newkeytheorem { mythm \int_use:N \l_my_style_int }[style=mysty \int_use:N \l_my_style_int]
	}

\int_new:N \l_my_thm_int
\clist_new:N \l_my_thm_clist
\clist_set:Nn \l_my_thm_clist {
	parent=theorem,
	sibling=theorem,
	title=SomeCrazyTitle,
	numbered=no,
	preheadhook={\color{blue}PREHEAD},
	postheadhook={\color{green}POSTHEAD},
	prefoothook={\color{red}PREFOOT},
	postfoothook={\color{yellow}POSTFOOT},
    shaded,
    thmbox,
	}
\clist_map_inline:Nn \l_my_thm_clist {
	\int_incr:N \l_my_thm_int
	\exp_args:Ne \newkeytheorem { mytestthm \int_use:N \l_my_thm_int }[#1]
	}
	
\NewDocumentCommand{\PrintTheorems}{}{
	\int_step_inline:nnn { 1 } { \int_use:N \l_my_style_int }
		{ \begin{mythm##1}[heading] \kant[2][1] \end{mythm##1} }
	\int_step_inline:nnn { 1 } { \int_use:N \l_my_thm_int }
			{ \begin{mytestthm##1}[heading] \kant[2][1] \end{mytestthm##1} }
	}
\ExplSyntaxOff

\newcommand{\pkg}{\textsf}