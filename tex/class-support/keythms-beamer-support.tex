% Maintained by Matthew Bertucci, 2024-present
% Please report all issues and feature requests at https://github.com/mbertucci47/keytheorems
% This work is licensed under the LPPL version 1.3c or later: https://www.latex-project.org/lppl.txt
\ProvidesExplFile{keythms-beamer-support}{\@keythms@date}{\@keythms@version}
  {keytheorems~support~for~the~beamer~class}

% do nothing if noamsthm loaded
\IfClassLoadedWithOptionsT{ beamer }{ noamsthm }{ \file_input_stop: }

\keys_define:nn { keytheorems/thmstyle }
  {
    inherit-style / example .meta:n =
      {
        bodyfont    = \normalfont,
        preheadhook = \def\inserttheoremblockenv{exampleblock},
      }
  }
\keys_define:nn { keytheorems/thm }
  { % redefine key to include \thm@headsep
    tcolorbox-no-titlebar .meta:n =
      {
        tcolorbox={
          notitle,
          before~upper={
            \group_begin:
            \usebeamerfont*{block title}
            \__keythms_thm_tcboxtemphead:
            \hskip\thm@headsep
            \group_end:
            },
          #1
          }
      },
  }
\cs_set_protected:Npn \__keythms_thm_tcboxcode:nn #1#2
  { % #1 = name, #2 = tcolorbox keys
    \RequirePackage{tcolorbox}
    \tcbset
      {
        keythms_tcbox_#1/.style =
          {
            savedelimiter=#1,
            #2
          }
      }
    \tl_gput_left:cn { g__keythms_thm_preheadfromkeys_#1_tl }
      {
        \cs_set_protected:Npn \thm@space@setup { \thm@preskip=0pt \thm@postskip=0pt }
        \cs_set:Npn \inserttheoremblockenv { keythms_beamer_tcbox }
      }
  }
\NewDocumentEnvironment{ keythms_beamer_tcbox }{ m d<> }
  {
    \cs_set:Npn \__keythms_thm_tcboxtemphead: { #1 } % for tcolorbox-no-titlebar
    \tl_if_novalue:nF { #2 } { \begin{onlyenv}<#2> }
    \begin{tcolorbox}[
      title={\usebeamerfont*{block title} #1},
      keythms_tcbox_\l_keythms_thmuse_envname_tl
      ]
  }
  {
    \end{tcolorbox}
    \tl_if_novalue:nF { #2 } { \end{onlyenv} }
  }

\cs_set_protected:Npn \keythms_keyify_theorem:n #1
  { % #1 = theorem name
    \DeclareEnvironmentCopy { keythms_orig_#1 } { #1 }
    \DeclareDocumentEnvironment { keythms_beamer_grab_#1 } { m m m +b }
      { % ##1 = keys, ##2 = note, ##3 = action spec, ##4 = theorem body
        \tl_if_empty:nTF { ##3 }
          { \begin{keythms_orig_#1}[{##2}] }
          { \begin{keythms_orig_#1}[{##2}]<##3> }
        \clist_map_inline:Nn \g__keythms_restatecounters_clist
          {
            \prop_gput:Nne \g__keythms_thmuse_othercounters_prop { ####1 }
              { \the\value{####1} }
          }
        \__keythms_thm_posthead_code:n { #1 }
        % below needs to come after posthead so that correct \@currentHref
        % is stored for tcolorbox theorems
        \__keythms_thm_addcontentsdata:nnnn { #1 }
          { \prop_to_keyval:N \g__keythms_thmuse_othercounters_prop }
          { ##1 } { ##4 }
        \__keythms_thm_tempstorerestatedata:nnn { #1 } { ##1 } { ##4 }
        ##4
        \__keythms_thm_prefoot_code:n { #1 }
        \end{keythms_orig_#1}
        \__keythms_thm_postfoot_code:n { #1 }
      }
      {}
    \DeclareDocumentEnvironment { keythms_beamer_grabreversed_#1 } { m m m +b }
      { % ##1 = keys, ##2 = note, ##3 = action spec, ##4 = theorem body
        \tl_if_exist:cTF
          { c__keythms_storeatbegin_ \l__keythms_thmuse_storereversed_tl _restatecounters_tl }
          {
            \bool_set_true:N \l__keythms_thmuse_restating_bool
            \prop_set_from_keyval:Ne \l__keythms_restate_counters_prop
              { \tl_use:c { c__keythms_storeatbegin_ \l__keythms_thmuse_storereversed_tl _restatecounters_tl } }
            \prop_map_inline:Nn \l__keythms_restate_counters_prop
              {
                \tl_set:ce { l_keythms_restate_current_####1_tl }
                  { \the\value{####1} }
                \setcounter { ####1 } { ####2 }
                % ^ FIX: what if eq's numbered by section, theorem, etc.? The
                %        thmtools code is opaque.... Or maybe should be up to the
                %        user to say "restate-counters={section,chapter,...}".
                \cs_set:cpn { theH ####1 }
                  { \use:c { the ####1 } . \theHkeythms_restate_dummyctr }
              }
            \tl_if_empty:cTF
              { c__keythms_storeatbegin_ \l__keythms_thmuse_storereversed_tl _label_tl }
              { \refstepcounter{keythms_restate_dummyctr} } % for unnumbered theorems
              {
                \cs_set:cpn { the #1 }
                  { \tl_use:c { c__keythms_storeatbegin_ \l__keythms_thmuse_storereversed_tl _label_tl } }
                \cs_set_eq:cN { c@ #1 } \c@keythms_restate_dummyctr
                \cs_set_eq:cN { theH #1 } \theHkeythms_restate_dummyctr
                % ^ why are the last two lines here? We shouldn't be referencing
                %   restated theorems. Think it's a remnant of thmtools
                % WRONG: needed to make numbering correct after restated theorem.
                % not sure about theH. <- this is needed to prevent duplicate anchors
              }
           }
           {
             \msg_warning:nnV { keytheorems } { store-reversed-not-got }
               \l__keythms_thmuse_storereversed_tl
           }
        \hook_use:n { keytheorems/#1/restated }
        \hook_use:n { keytheorems/allthms/restated }
        \tl_if_empty:nTF { ##3 }
          { \begin{keythms_orig_#1}[{##2}] }
          { \begin{keythms_orig_#1}[{##2}]<##3> }
        \__keythms_thm_posthead_code:n { #1 }
        % below needs to come after posthead so that correct \@currentHref
        % is stored for tcolorbox theorems
        \__keythms_thm_addstoredreverseddata:nnn { #1 } { ##1 } { ##4 }
        \__keythms_thm_tempstorerestatedatareversed:nnn { #1 } { ##1 } { ##4 }
        ##4
        \__keythms_thm_prefoot_code:n { #1 }
        \end{keythms_orig_#1}
        \__keythms_thm_postfoot_code:n { #1 }
        \prop_map_inline:Nn \l__keythms_restate_counters_prop
          {
            \exp_args:Nnc \setcounter { ####1 }
              { l_keythms_restate_current_####1_tl }
          }
      }
      { }
      % NOTE: have to do a lot of shenanigans to make sure the begin/end of grabbed
      %       theorem env captures only the body and no package code.
      %       This is the price of on-the-fly redefining the env to grab body
      \RenewDocumentEnvironment { #1 } { D<>{} ={note} O{} D<>{} }
        {
          \tl_set:Nn \l_keythms_thmuse_envname_tl { #1 }
          \keys_set:nn { keytheorems/thmuse } { ##2 }
          \tl_if_empty:NF \l__keythms_thmuse_store_tl
            {
              % \bool_gset_true:N \g__keythms_listof_writefile_bool % disable writing to file
              \cs_set_eq:NN \__keythms_beamer_withhooks_begin:nnnn \__keythms_beamer_grab_begin:nnnn
              \cs_set_eq:NN \__keythms_beamer_withhooks_end:n \__keythms_beamer_grab_end:n
            }
          \tl_if_empty:NF \l__keythms_thmuse_storereversed_tl
            {
              % \bool_gset_true:N \g__keythms_listof_writefile_bool % disable writing to file
              \cs_set_eq:NN \__keythms_beamer_withhooks_begin:nnnn \__keythms_beamer_grabreversed_begin:nnnn
              \cs_set_eq:NN \__keythms_beamer_withhooks_end:n \__keythms_beamer_grabreversed_end:n
            }
          \__keythms_thm_prehead_continues_code:n { #1 }
          \__keythms_thm_prehead_code:n { #1 }
          \tl_if_empty:nTF { ##3 }
            { % if ##3 empty, just pass ##1 and withhooks_begin will check if empty
              \__keythms_beamer_withhooks_begin:nnVn { #1 } { ##2 }
                \l__keythms_thmuse_note_tl { ##1 }
            }
            { % if ##3 nonempty, always use it (beamer defaults to second <> if both given)
              \__keythms_beamer_withhooks_begin:nnVn { #1 } { ##2 }
                \l__keythms_thmuse_note_tl { ##3 }
            }
        }
        {
          \__keythms_beamer_withhooks_end:n { #1 }
          \tl_if_empty:NF \l__keythms_thmuse_store_tl
            {
              \cs_if_exist:cF
                { __keythms_getthm_ \l__keythms_thmuse_store_tl _theorem }
                {
                  \cs_new:cpe
                    { __keythms_getthm_ \l__keythms_thmuse_store_tl _theorem }
                    {
                      \exp_not:N \__keythms_getthm_theorem:nnnnn
                      \exp_not:o { \g__keythms_thmuse_temprestatedata_tl }
                    }
                  \cs_new:cpe
                    { __keythms_getthm_ \l__keythms_thmuse_store_tl _body }
                    {
                      \exp_not:N \__keythms_getthm_body:nn
                      \exp_args:No \exp_not:o
                        {
                          \exp_after:wN \__keythms_use_iii_v_braced:nnnnn
                            \g__keythms_thmuse_temprestatedata_tl
                        }
                    }
                }
            }
          \tl_if_empty:NF \l__keythms_thmuse_storereversed_tl
            {
              \cs_if_exist:cF
                { __keythms_getthm_ \l__keythms_thmuse_storereversed_tl _theorem }
                {
                  \cs_new:cpe
                    { __keythms_getthm_ \l__keythms_thmuse_storereversed_tl _theorem }
                    {
                      \exp_not:N \__keythms_getthmreversed_theorem:nnn
                      \exp_not:o { \g__keythms_thmuse_temprestatedatareversed_tl }
                    }
                  \cs_new:cpn
                    { __keythms_getthm_ \l__keythms_thmuse_storereversed_tl _body }
                    {
                      \textbf{??}
                      \msg_warning:nnn { keytheorems } { restate-body-never-got }
                        { #1 }
                    }
                }
            }
        }
  }

\cs_new_protected:Npn \__keythms_beamer_withhooks_begin:nnnn #1#2#3#4
  { % #1 = theorem name, #2 = keys, #3 = note, #4 = action spec
    \tl_if_empty:nTF { #4 }
      { \begin{keythms_orig_#1}[{#3}] }
      { \begin{keythms_orig_#1}[{#3}]<#4> }
    \__keythms_thm_posthead_code:n { #1 }
    \__keythms_thm_addcontentsdata:nnnn { #1 } { } { #2 } { }
    \ignorespaces % I hope this is alright
  }
\cs_generate_variant:Nn \__keythms_beamer_withhooks_begin:nnnn { nnV }
\cs_new_eq:NN \__keythms_beamer_withhooks_end:n \__keythms_withhooks_end:n

\cs_new_protected:Npn \__keythms_beamer_grab_begin:nnnn #1#2#3#4
  { % #1 = theorem name, #2 = keys, #3 = note
    \begin{keythms_beamer_grab_#1}{#2}{#3}{#4}
  }
\cs_generate_variant:Nn \__keythms_beamer_grab_begin:nnnn { nnV }
\cs_new_protected:Npn \__keythms_beamer_grab_end:n #1 { \end{keythms_beamer_grab_#1} }

\cs_new_protected:Npn \__keythms_beamer_grabreversed_begin:nnnn #1#2#3#4
  { % #1 = theorem name, #2 = keys, #3 = note
    \begin{keythms_beamer_grabreversed_#1}{#2}{#3}{#4}
  }
\cs_generate_variant:Nn \__keythms_beamer_grabreversed_begin:nnnn { nnV }
\cs_new_protected:Npn \__keythms_beamer_grabreversed_end:n #1
  { \end{keythms_beamer_grabreversed_#1} }
  
% for now, just disable writing to file
\cs_set_eq:NN \__keythms_thm_addcontentsdata:nnnn \use_none:nnnn
\cs_set_eq:NN \__keythms_thm_addstoredreverseddata:nnn \use_none:nnn

% disable \listofkeytheorems, add warning
\msg_new:nnn { keytheorems } { beamer-listof }
  {
    \protect\listofkeytheorems\space not~supported~with~beamer
  }
\RenewDocumentCommand \listofkeytheorems { o }
  {
    \msg_warning:nn { keytheorems } { beamer-listof }
  }

% restatable and restatable* environments need redefinition
\bool_if:NT \g__keythms_thmtoolscompat_bool
  {
    \RenewDocumentEnvironment { restatable } { O{} m m }
      {
        % set store outside [] so keyless note is recognized
        \keys_set:nn { keytheorems/thmuse } { store=#3 }
        \begin{#2}[#1]
      }
      {
        \end{#2}
        \cs_gset_protected:cpn { #3 }
          { % make \foo and \foo* identical
            \peek_meaning_remove:NTF *
              { \getkeytheorem{ #3 } }
              { \getkeytheorem{ #3 } }
          }
      }
    \RenewDocumentEnvironment { restatable* } { O{} m m }
      {
        % set store* outside [] so keyless note is recognized
        \keys_set:nn { keytheorems/thmuse } { store*=#3 }
        \begin{#2}[#1]
      }
      {
        \end{#2}
        \cs_gset_protected:cpn { #3 }
          { % make \foo and \foo* identical
            \peek_meaning_remove:NTF *
              { \getkeytheorem{ #3 } }
              { \getkeytheorem{ #3 } }
          }
      }
  }

\file_input_stop: